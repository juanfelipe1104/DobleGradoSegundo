\documentclass[fleqn]{article}
\usepackage{amsmath}
\usepackage{amsfonts}

\author{Juan Rodríguez}
\title{Formulario Memoria Virtual Sistemas Operativos Práctico}
\date{12/01/2025}

\begin{document}
	\maketitle
	Todo se expresa en potencias de 2
	\section{Memoria virtual}
	Una memoria virtual se compone de número de página + offset \\
	
	Una memoria física se compone de número de marco + offset \\
	
	Tamaño de la página $(2^{o})$ $=$ Tamaño del marco $(2^{r})$ $\rightarrow$ $o$ $=$ $r$ \\
	
	Se reservan $o$ bits para el offset \\
	
	Número de páginas $(2^{p})$ $=$ $\frac{\text{Tamaño de la memoria virtual $(2^v)$}}{\text{Tamaño página$(2^{o})$}}$ \\
	
	Se reservan $p$ bits para la página \\
	
	Número de marcos $(2^{m})$ $=$ $\frac{\text{Tamaño de la memoria física $(2^f)$}}{\text{Tamaño marco $(2^{r})$}}$ \\
	
	Se reservan $m$ bits para el marco
	\section{Algoritmos de reemplazo}
	FIFO: Si una página que estaba cargada entra, se mantiene la cola igual. Si no está cargada, sale la página que lleva más tiempo (La que primero entró - First In First Out) \\
	
	LRU: Si una página que estaba cargada entra, se envía al final de la cola. Si no está cargada, sale la página que lleva más tiempo \\
	
	Second Chance: Si una página que estaba cargada entra, se le pone un $*$ (Segunda vida). Si no está cargada, sale la página que lleva más tiempo; pero si la página es una página con $*$, se envía al final la página con $*$ ("Se salva de ser eliminada")
\end{document}