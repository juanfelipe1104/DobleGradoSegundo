\documentclass[fleqn]{article}
\usepackage{amsmath}
\usepackage{amsfonts}
\title{Solución Problemas Números Complejos}
\author{Juan Rodríguez}

\setlength{\parindent}{0pt}

\begin{document}
	\maketitle
	\section*{Apuntes}
	\begin{itemize}
		\item $i^{2} = -1$
		\item $i^{3} = i^{2} \cdot i = -i$
		\item $i^{4} = i^{2} \cdot i^{2} = 1$
		\item $(a+bi) \cdot (c+di) = (ac-bd+adi+bci)$
		\item $\frac{a+bi}{c+di} = \frac{a+bi}{c+di} \cdot \frac{c-di}{c-di} = \frac{(a+bi) \cdot (c-di)}{c^{2}+d^{2}}$
		\item $x = e^{Ln(x)} \leftrightarrow x = Ln(e^{x})$
	\end{itemize}
	\section{Ejercicio 1}
	Calcular parte real e imaginaria de los siguientes números complejos: \\
	a) $z_1 = \frac{3-2i}{1+4i} $
	b) $z_2 = \frac{1}{i} + \frac{1}{1+i}$
	c) $z_3 = cos(i)$
	d) $z_4 = sen(2+i)$
	\subsection{Solución}
	\[
	z_1 = \frac{3-2i}{1+4i} = \frac{3-2i}{1+4i} \cdot \frac{1-4i}{1-4i} = \frac{3-8-12i-2i}{17} = \boxed{-\frac{5}{17} - \frac{14}{17}i}
	\]
	\[
	z_2 = \frac{1}{i} + \frac{1}{1+i} = \frac{1+2i}{-1+i} = \frac{1+2i}{-1+i} \cdot \frac{-1-i}{-1-i} = \frac{-1+2-1i-2i}{2} = \boxed{\frac{1}{2} - \frac{3}{2}i}
	\]
	\[
	z_3 = cos(i) = \frac{e^{-1} + e}{2} = \frac{\frac{1}{e}+e}{2} = \boxed{\frac{1}{2e}+\frac{e}{2}}
	\]
	\[
	z_4 = sen(2+i) = \frac{e^{i(2+i)}-e^{-i(2+i)}}{2i} = \frac{e^{-1+2i}-e^{1-2i}}{2i} = -\frac{i}{2} (\frac{1}{e} \cdot (cos(2)+isen(2)) - e \cdot (cos(-2)+isen(-2)))
	\]
	\[
	= -\frac{i}{2} (\frac{1}{e} \cdot (cos(2)+isen(2)) - e \cdot (cos(2)-isen(2))) = \frac{1}{2e} (sen(2)-icos(2)) - \frac{e}{2} (-sen(2)-icos(2))
	\]
	\[
	= \boxed{(\frac{1}{2e}+\frac{e}{2})sen(2) + i(-\frac{1}{2e}+\frac{e}{2})cos(2)}
	\]
	\section{Ejercicio 2}
	Calcular el modulo y argumento de: \\
	a) $z_1 = 3^i$
	b) $z_2 = i^{i}$
	c) $z_3 = i^{3+i}$
	d) $z_4 = (1+i)^{2+i}$
	\subsection{Solución}
	\[
	z_1 = 3^{i} = e^{Ln(3^{i})} = e^{i Ln(3)} = \boxed{r=1 , \sigma = Ln (3)}
	\]
	\[
	z_2 = (e^{i\frac{\pi}{2}})^{i} = e^{-\frac{\pi}{2}} = \boxed{r=e^{-\frac{\pi}{2}}, \sigma = 0}
	\]
	\[
	z_3 = i^{3+i} = i^{3} \cdot i^{i} = (-i) \cdot e^{-\frac{\pi}{2}} = e^{-\frac{\pi}{2}} \cdot e^{i\frac{3\pi}{2}} = \boxed{r=e^{-\frac{\pi}{2}}, \sigma = \frac{\pi}{2}}
	\]
	\[
	z_4 = (1+i)^{2+i} = (1+i)^{2} \cdot (1+i)^{i} = (\sqrt{2} \cdot e^{i\frac{\pi}{4}})^{2} \cdot (\sqrt{2} \cdot e^{i\frac{\pi}{4}})^{i} = (2 \cdot e^{i\frac{\pi}{2}}) \cdot ((\sqrt{2})^{i} \cdot e^{-\frac{\pi}{4}})
	\]
	\[
	= 2e^{-\frac{\pi}{4}} \cdot e^{i\frac{\pi}{2}} \cdot e^{iLn(\sqrt{2})} = 2e^{-\frac{\pi}{4}} \cdot e^{i(\frac{\pi}{2} + Ln(\sqrt{2}))} = \boxed{r=2e^{-\frac{\pi}{4}}, \sigma = \frac{\pi}{2} + Ln(\sqrt{2})}
	\]
	\section{Ejercicio 3}
	Calcular las raíces cuartas de la unidad
	\subsection{Solución}
	\[
	z^{4} = 1 \rightarrow z = \sqrt[4]{1} e^{i(\frac{2k\pi}{4})} || k = 0,1,2,3
	\]
	\[
	k = 0 \rightarrow e^{0} = \boxed{1}
	\]
	\[
	k = 1 \rightarrow e^{i\frac{\pi}{2}} = \boxed{i}
	\]
	\[
	k = 2 \rightarrow e^{i\pi} = \boxed{-1}
	\]
	\[
	k = 3 \rightarrow e^{i3\frac{\pi}{2}} = \boxed{-i}
	\]
	\section{Ejercicio 4}
	Calcular las raíces cúbicas de -8
	\subsection{Solución}
	\[
	z^{3} = -8 \rightarrow \sqrt[3]{8} e^{i(\frac{\pi + 2k\pi}{3})} || k = 0,1,2
	\]
	\[
	k = 0 \rightarrow 2e^{i\frac{\pi}{3}} = 2(cos(\frac{\pi}{3}) + isen(\frac{\pi}{3})) = \boxed{1 + i\sqrt{3}} 
	\]
	\[
	k = 1 \rightarrow 2e^{i\pi} = \boxed{-2}
	\]
	\[
	k = 2 \rightarrow 2e^{i\frac{5\pi}{3}} = 2(cos(\frac{5\pi}{3}) + isen(\frac{5\pi}{3})) = \boxed{1 - i\sqrt{3}}
	\]
	\section{Ejercicio 5}
	Determina la forma binómica de $e^{\sqrt{i}}$
	\subsection{Solución}
	\[
	\sqrt{i} = \sqrt{e^{i\frac{\pi}{2}}} = e^{i(\frac{\frac{\pi}{2} + 2k\pi}{2})} || k = 0,1
	\]
	\[
	k = 0 \rightarrow e^{i\frac{\pi}{4}} = cos(\frac{\pi}{4}) + sen(\frac{\pi}{4}) = \frac{\sqrt{2}}{2} + i\frac{\sqrt{2}}{2}
	\]
	\[
	k = 1 \rightarrow e^{i\frac{5\pi}{4}} = cos(\frac{5\pi}{4}) + sen(5\frac{\pi}{4}) = -\frac{\sqrt{2}}{2} - i\frac{\sqrt{2}}{2}
	\]
	Opción 1
	\[
	e^{\sqrt{i}} = e^{\frac{\sqrt{2}}{2} + i\frac{\sqrt{2}}{2}} = e^{\frac{\sqrt{2}}{2}}(cos(\frac{\sqrt{2}}{2}) + i sen(\frac{\sqrt{2}}{2})) = \boxed{e^{\frac{\sqrt{2}}{2}}cos(\frac{\sqrt{2}}{2}) + ie^{\frac{\sqrt{2}}{2}}sen(\frac{\sqrt{2}}{2})}
	\]
	Opción 2
	\[
	e^{\sqrt{i}} = e^{\frac{-\sqrt{2}}{2} - i\frac{\sqrt{2}}{2}} = e^{\frac{-\sqrt{2}}{2}}(cos(-\frac{\sqrt{2}}{2}) + i sen(-\frac{\sqrt{2}}{2})) = \boxed{e^{-\frac{\sqrt{2}}{2}}cos(\frac{\sqrt{2}}{2}) - ie^{-\frac{\sqrt{2}}{2}}sen(\frac{\sqrt{2}}{2})}
	\]
	\section{Ejercicio 6}
	Expresa en forma binómica el número $(-\frac{\sqrt{3}}{2} + \frac{i}{2})^{6}$
	\subsection{Solución}
	\[
	(-\frac{\sqrt{3}}{2} + \frac{i}{2})^{6} = (e^{i\frac{5\pi}{6}})^{6} = e^{i5\pi} = e^{i\pi} = \boxed{-1}
	\]
	\section{Ejercicio 7}
	Calcula $\sqrt{-16-30i}$
	\subsection{Solución}
	\[
	\sqrt{-16-30i} = (x+iy) \rightarrow (\sqrt{-16-30i})^{2} = (x+iy)^{2} \rightarrow -16-30i = x^{2} + 2ixy - y^{2} \rightarrow
	\]
	\[
	x^{2} - y^{2} = -16
	\]
	\[
	2xy = -30 \rightarrow x = -\frac{15}{y}
	\]
	\[
	(-\frac{15}{y})^{2} - y^{2} = -16 \rightarrow \frac{225}{y^{2}} - y^{2} = -16 \rightarrow y^{4} +16y^{2} - 225 = 0
	\]
	\[
	\underrightarrow{y^{2} = t}
	\]
	\[
	t^{2} +16t -225 = 0 \rightarrow (t-25)(t+9) = 0 \rightarrow y=\pm5, x=\mp3
	\]
	\[
	\boxed{z_1 = -3 + 5i, z_2 = 3 -5i}
	\]
	\section{Ejercicio 8}
	Determinar $m \in \mathbb{R}$ de modo que $(2e^{i\sqrt{2}})^{m}$ sea un número real negativo
	\subsection{Solución}
	\[
	(2e^{i\sqrt{2}})^{m} = 2^{m} \cdot \underline{e^{im\sqrt{2}}}
	\]
	Nota: Nos fijamos en la parte imaginaria ya que $2^{m}$ siempre sera un valor positivo
	\[
	2^{m}(cos(m\sqrt{2}) + isen(m\sqrt{2}))
	\]
	Nota: Para que el número sea real negativo, el coseno debe ser negativo y el seno nulo. Por lo tanto, $\forall k \in \mathbb{Z} \rightarrow cos((2k+1)\pi) = -1, sen((2k+1)\pi) = 0$
	\[
	m\sqrt{2} = (2k+1)\pi \rightarrow \boxed{m = \frac{(2k+1)\pi}{\sqrt{2}} || k \in \mathbb{Z}}
	\]
	\section{Ejercicio 9}
	Determinar los números complejos no nulos tal que su quinta potencia, $z^{5}$, sea igual a su conjugado, es decir, $\overline{z}$
	\subsection{Solución}
	\[
	z = re^{i\sigma}, \overline{z} = z = re^{-i\sigma}
	\]
	\[
	(re^{i\sigma})^{5} = re^{-i\sigma} \rightarrow r^{5}e^{i\sigma5} = re^{-i\sigma}
	\]
	\[
	r^{5} = r \rightarrow r(r^{4} - 1) = 0 \rightarrow r = 0 X, r = -1 X, r = 1 \checkmark
	\]
	\[
	e^{i\sigma5} = e^{-i\sigma} \rightarrow 5\sigma = -\sigma + 2k\pi \rightarrow \sigma = \frac{k\pi}{3}
	\]
	\[
	k = 0 \rightarrow z = 1, \boxed{1^{5} = \overline{1}}
	\]
	\[
	k = 1 \rightarrow z = e^{i\frac{\pi}{3}}, \boxed{e^{i\frac{5\pi}{3}} = e^{-i\frac{\pi}{3}}}
	\]
	\[
	\boxed{\forall k \in \mathbb{N} \rightarrow (e^{i\frac{k\pi}{3}})^{5} = e^{-i\frac{k\pi}{3}}}
	\]
	\section{Ejercicio 10}
	Dado el polinomio $P(z) = z^{4} -6z^{3} + 24z^{2} -18z + 63$. Calcula $P(i\sqrt{3})$ y $P(-i\sqrt{3})$. Resuelve a continuación la ecuación $P(z) = 0$
	\subsection{Solución}
	\[
	P(i\sqrt{3}) = (i\sqrt{3})^{4} - 6(i\sqrt{3})^{3} + 24 (i\sqrt{3})^{2} - 18(i\sqrt{3}) + 63 = 9 + 18\sqrt{3}i - 72 - 18\sqrt{3}i + 63 = \boxed{0}
	\]
	\[
	\boxed{P(-i\sqrt{3}) = 0} 
	\]
	Nota: Si un número complejo es solución o raíz de un polinomio de coeficientes reales, su conjugado también lo es. \\
	\[
	P(z) = (z - i\sqrt{3})(z - (-i\sqrt{3})) \cdot Q(x) = (z^{2} + 3) \cdot Q(x) 
 	\]
 	\[
 	Q(x) = \frac{z^{4} -6z^{3} + 24z^{2} -18z + 63}{z^{2} + 3} = \text{...} = z^{2} - 6z + 21 = \text{...} = (z - (3+2\sqrt{3}i))(z - (3-2\sqrt{3}i))
 	\]
 	\[
 	P(z) = 0 \rightarrow \boxed{P(z) = (z - i\sqrt{3})(z - (-i\sqrt{3}))(z - (3+2\sqrt{3}i))(z - (3-2\sqrt{3}i))}
 	\]
 	\section{Ejercicio 11}
 	Dado el número complejo $z = -\sqrt{2+\sqrt{2}} + i\sqrt{2-\sqrt{2}}$, calcula $z^{2}$ en forma binómica. A continuación, expresa $z^{2}$ en forma exponencial y deduce la forma exponencial de $z$
 	\subsection{Solución}
 	 \[
 	 z^{2} = (-\sqrt{2+\sqrt{2}} + i\sqrt{2-\sqrt{2}})^{2} = 2 + \sqrt{2} - 2 + \sqrt{2} - 2i(\sqrt{2+\sqrt{2}})(\sqrt{2-\sqrt{2}}) = \boxed{2\sqrt{2} - 2i\sqrt{2}}
 	 \]
 	 \[
 	 \begin{cases}
 	 	r = \sqrt{(2\sqrt{2})^{2} + (2\sqrt{2})^{2}} = 4 \\
 	 	\sigma = arctg(\frac{-2\sqrt{2}}{2\sqrt{2}}) = -\frac{\pi}{4} = \frac{7\pi}{4}
 	 \end{cases}
 	 \]
 	 \[
 	 z^{2} = 4e^{\frac{7\pi}{4}}
 	 \]
 	 \[
 	 z = \sqrt{4e^{\frac{7\pi}{4}}} = 2e^{i\frac{\frac{7\pi}{4}+ 2k\pi}{2}} \text{ tal que } k = 0,1
 	 \]
 	 \[
 	 \begin{cases}
 	 	k = 0 & 2e^{i\frac{7\pi}{8}} \text{ Correcto, segundo cuadrante} \\
 	 	k = 1 & 2e^{i\frac{15\pi}{8}} \text{ Incorrecto, cuarto cuadrante}
 	 \end{cases}
 	 \]
 	 Nota: El número original estaba en el segundo cuadrante. Al elevar al cuadrado y después deshacerlo, podemos introducir soluciones irreales.
\end{document}